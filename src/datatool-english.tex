% arara: pdflatex
\documentclass{ltxdoc}

\usepackage{fontspec}
\usepackage{tcolorbox}

\setromanfont{Noto Serif}
\setsansfont{Noto Sans}
\setmonofont{Noto Sans Mono}
\newfontfamily\runefont{Noto Sans Runic}
\NewDocumentCommand{\textrune}{m}{{\runefont #1}}

\CheckSum{0}

\newcommand*{\sty}[1]{\textsf{#1}}
\newcommand*{\file}[1]{\nolinkurl{#1}}
\newcommand*{\filemeta}[3]{\nolinkurl{#1}\meta{#2}\nolinkurl{#3}}
\newcommand*{\filemetameta}[5]{%
 \nolinkurl{#1}\meta{#2}\nolinkurl{#3}\meta{#4}\nolinkurl{#5}}
\newcommand*{\opt}[1]{\textsf{#1}}
\newcommand*{\qt}[1]{``#1''}

\definecolor{defbackground}{rgb}{1,1,0.75}
\newtcolorbox{definition}{colback=defbackground}
\newtcolorbox{important}{colback=red!5!white,colframe=red}

\tcbuselibrary{documentation}
\tcbset{verbatim ignore percent}

\RecordChanges
\PageIndex

\title{English Localisation Support for \sty{datatool} Package}
\author{Nicola L. C. Talbot}
%%DATECMD%%

\begin{document}
\maketitle

\begin{abstract}
This is the English localisation support for the \sty{datatool}
package (version 3.0+). This needs to be installed in addition to 
\sty{datatool}. To ensure regional support, you will also need to
install \sty{datatool-regions}.
\end{abstract}

\section{Introduction}
\label{sec:intro}

This bundle provides the English modules for \sty{datatool} v3.0+.
The files simply need to be installed on \TeX's path.
The \sty{datatool-base} package (which is automatically loaded by
\sty{datatool}) uses \sty{tracklang}'s interface for detecting
localisation settings and finding the appropriate files.
If you use \sty{babel} or \sty{polyglossia}, make sure that you
specify the document languages before the first package to load
\sty{tracklang}.

For example:
\begin{dispListing}
\usepackage[british]{babel}
\usepackage{datatool-base}
\end{dispListing}
Alternatively, if you are not using a language package, simply use
the \opt{locales} option. For example:
\begin{dispListing}
\usepackage[locales={en-GB}]{datatool-base}
\end{dispListing}
Any option that can be passed to \sty{datatool-base} can also be
passed to \sty{datatool} but if \sty{datatool-base} has already been
loaded, it will be too late to use the \opt{locales} option.
For example:
\begin{dispListing}
\usepackage[locales={en-GB}]{datatool}
\end{dispListing}
But not:
\begin{dispListing*}{title={Incorrect!},colframe=red}
\usepackage{datatool-base}
\usepackage[locales={en-GB}]{datatool}
\end{dispListing*}

If another package that also loads \sty{tracklang} is loaded first,
then \sty{datatool-base} can pick up the settings from that. For
example:
\begin{dispListing}
\usepackage[en-GB]{datetime2}
\usepackage{datatool}
\end{dispListing}

If the regional file isn't installed or if no region is associated
with the locale then only the language settings will be implemented.
For example:
\begin{dispListing}
\usepackage[english]{babel}
\usepackage{datatool-base}
\end{dispListing}
In this case there is no region so only \file{datatool-english.ldf}
will be loaded.

Supplementary packages provided with \sty{datatool} can also have
the locales provided. For example:
\begin{dispListing}
\documentclass{article}
\usepackage[locales={en}]{datagidx}
\newgidx{index}{Index}
\newterm{élan}
\newterm{elephant}
\newterm{élite}
\newterm{elk}
\begin{document}
\gls{elk}, \gls{élan} \gls{élite} and \gls{elephant}.
\printterms
\end{document}
\end{dispListing}

The \sty{glossaries} package also loads \sty{datatool-base}
and as from version 4.55 will check for the new \sty{datatool-base}
commands. If present, \cs{printnoidxglossary} will switch to this
new method of sorting otherwise it will fallback on the old method.
So installing \sty{datatool-english} will not only affect
\sty{datatool} and its associated packages but also
\sty{glossaries}. For example:
\begin{dispListing}
\documentclass{article}
\usepackage[locales={en},index]{glossaries}
\newterm{élan}
\newterm{elephant}
\newterm{élite}
\newterm{elk}
\begin{document}
\gls{elk}, \gls{élan} \gls{élite} and \gls{elephant}.
\printindex
\end{document}
\end{dispListing}

\begin{docCommand}{DataToolBaseEnglish}{}
The \file{datatool-english.ldf} file provides an intermediary command
\cs{DataToolBaseEnglish} that's automatically added to the captions hook.
This performs all the \sty{datatool-base} English language redefinitions.
There should be little reason to actually use this command anywhere.
\end{docCommand}

\section{Orthography}
\label{sec:orthography}

The \file{datatool-english.ldf} file provides support for English
orthography. This deals with how words are sorted according to the
English alphabet. The way that the new \sty{datatool} v3.0 sorting
commands work is to use a handler function that converts the
original content into a byte string. Since there are no byte array
data types in \TeX, this is implemented by converting the original
string into an ASCII string in such a way that sorting by character
code will produce the desired order.

With the newer \cs{DTLsortwordlist}, this is processing is performed once for each
item in the list prior to sorting to allow for a faster character code
sort of a temporary sequence variable that stores both the original
(actual value) and the transformed (sort value) item. 
This allows the original item to be restored afterwards.

For example:
\begin{dispListing*}{title={\sty{datatool} v3.0+}}
\documentclass{article}
\usepackage[locales={en}]{datatool}
\newcommand{\mylist}{élan,zebra,elephant,ant,élite,elk}
\begin{document}
\DTLsortwordlist{\mylist}{\DTLsortwordcasehandler}
Sorted list: \DTLformatlist{\mylist}
\end{document}
\end{dispListing*}
If \file{datatool-english.ldf} is installed, then the resulting list
will be:
\begin{quote}
ant, élan, elephant, élite, elk and zebra
\end{quote}
However, if \file{datatool-english.ldf} isn't installed (or if
English support hasn't been requested) the resulting list would be:
\begin{quote}
ant, elephant, elk, zebra, élan \& élite
\end{quote}
This is because a simple character code has been used.
(Note also the use of \& instead of ``and''.)

The \cs{DTLsortwordcasehandler} function is for case-sensitive
sorts. The \cs{DTLsortwordhandler} function is for case-insensitive
sorts and simply converts each element to lowercase before
processing using the locale handler. These functions are for word-order
comparisons.
There are analogous functions for letter-order comparisons:
\cs{DTLsortletterhandler} (case-insensitive) and 
\cs{DTLsortlettercasehandler} (case-sensitive). These strip spaces
and hyphens before processing.

The \cs{DTLsortdata} command, provided by \sty{datatool}, works in a
similar way to \cs{DTLsortwordlist} and uses the same handler
functions. For example:
\begin{dispListing}
\documentclass{article}
\usepackage[locales={en}]{datatool}

\DTLaction{new}
\DTLaction
 [
   assign = { name = Annie, age = 40 }
 ]
{new row}

\DTLaction
 [
   assign = { name = Ele, age = 80 }
 ]
{new row}

\DTLaction
 [
   assign = { name = Éleanor, age = 28 }
 ]
{new row}

\DTLaction
 [
   assign = { name = Zack, age = 47 }
 ]
{new row}

\DTLaction
 [
   assign = { name = Elva, age = 53 }
 ]
{new row}

\DTLaction
 [
   assign = { name = Æthelwulf, age = 95 }
 ]
{new row}

\DTLsortdata{}{name}
\begin{document}
\DTLaction{display}
\end{document}
\end{dispListing}
If \file{datatool-english.ldf} is installed, then the order will be:
Æthelwulf, Annie, Ele, Éleanor, Elva, Zack.

Without \file{datatool-english.ldf}, the order will be:
Annie, Ele, Elva, Zack, Æthelwulf, Éleanor.

If letter groups are required, an extra column can be added to the
database with the corresponding letter group obtained from sorting:
\begin{dispListing}
\DTLsortdata[save-group]{}{name}
\end{dispListing}
The letter group is typically the first letter of the sort value
but this may not be the case for some languages, so the code for
obtaining the first letter of a word is adjusted by the language
hook.

The groups obtained from sorting are letter groups if the sorted
item starts with an alphabetical character. If the value is
determined to be a currency, it will be considered part of the
currency group, if the value is determined to be a number (without a
currency prefix) then it will be considered part of the number
group, otherwise it will be in the non-letter group.

The titles for all these groups are obtained with the commands
\cs{dtllettergroup}, \cs{dtlnonlettergroup}, \cs{dtlnumbergroup}
and \cs{dtlcurrencygroup} which are all redefined by the language
hook.

In the case of English, the currency group title simply expands to
\qt{Currency}, the number group title simply expands to
\qt{Numbers}, the non-letter group title simply expands to
\qt{Symbols}.  The letter group title expands to its argument
converted to title case.

If you need to make any adjustments, the following commands are
provided by \file{datatool-english.ldf} for the handler and 
groups.

\begin{docCommand}{DTLenLocaleHandler}{\marg{tl-var}}
Converts the token list variable containing the sort value.
You may redefine this command to add support for additional
characters.
\end{docCommand}

\begin{docCommand}{DTLenLocaleGetInitialLetter}{\marg{word}\marg{tl-var}}
Gets the initial letter of \meta{word} and stores it in the given
token list variable.
\end{docCommand}

\begin{docCommand}{DTLenSetLetterGroups}{}
Redefines the group commands. This can be redefined if different
titles are required.
\end{docCommand}

\section{Dates and Times}
\label{sec:temporal}

This is still an experimental feature. As from v3.0, \sty{datatool}
now has additional data types: date, time and datetime.
Parsing for these is off by default but may be enabled.
Without the language support, only ISO dates, times and timestamps
can be parsed. The order (dmy, mdy, ymd) is usually set by the
region file (provided with \sty{datatool-regions}).
For example, \file{datatool-GB.ldf} sets the order to dmy
whereas \file{datatool-US.ldf} sets the order to mdy.
The region files only deal with numeric dates and times but
have a hook to allow supporting language files to extend this to
parsing temporal values that contain textual content, such as month
names.

The following commands are provided. Note that they require
\LaTeX3 syntax.

\begin{docCommand*}{datatool_en_get_monthname_map:n}{ \marg{month}}
Gets the number associated with the given English month name.
\end{docCommand*}

\begin{docCommand*}{datatool_en_if_pm:nTF}{ \marg{text} \marg{true} \marg{false}}
Tests if \meta{text} represents afternoon (\qt{pm}, 
\qt{in the afternoon}, \qt{in the evening}, \qt{midnight}).
This indicates that 12 needs to be added to the hour for 12-hour
formats.
\end{docCommand*}

\begin{docCommand*}{datatool_en_monthname:n}{ \marg{num}}
Expands to the month name for \meta{num} (1 for January, etc).
\end{docCommand*}

\begin{docCommand*}{datatool_en_shortmonthname:n}{ \marg{num}}
Expands to the abbreviated month name. This will be defined to
\cs{DTMenglishshortmonthname} if that command exists
(\sty{datetime2-english})
or to:
\end{docCommand*}
\begin{docCommand*}{datatool_en_shortmonthname_dotless:n}{ \marg{num}}
Expands to a three-letter abbreviation without a trailing dot.
\end{docCommand*}
An alternative command is also provided:
\begin{docCommand*}{datatool_en_shortmonthname_dotted:n}{ \marg{num}}
This expands to a three or four letter abbreviation followed by a
dot except for short month names, such as May, that aren't
abbreviated.
\end{docCommand*}

\section{Support for \sty{databib}}

The \sty{datatool-english} bundle provides language support for
the \sty{databib} package. Options may be set with
\cs{DTLsetLocaleOptions}\verb|{en/databib}|\marg{key=val list}.

Currently, there is only one option: \opt{short-month-style}
which may take the value \opt{dotted} (use dotted month
abbreviations) and \opt{dotless} (use three letter abbreviations
with no dot). This determines the month name abbreviations for 
\sty{databib}'s \opt{abbrv} style.

\begin{docCommand}{DataBibEnglish}{}
The \file{databib-english.ldf} file defines the
intermediary command \cs{DataBibEnglish}
that redefines all the fixed text commands. This command
is automatically added to the captions hook if \sty{databib}
is loaded.
\end{docCommand}

\section{Support for \sty{person}}

\begin{docCommand}{DataToolPersonEnglish}{}
The \sty{datatool-english} bundle provides language support for
the \sty{person} package. There are currently no options.
The \file{person-english.ldf} file defines the intermediary
command \cs{DataToolPersonEnglish}
that redefines all the fixed text commands. This command
is automatically added to the captions hook if \sty{person}
is loaded.
\end{docCommand}

\section{Old English (Anglo-Saxon)}

This bundle also supplies very limited support for Old English
(Anglo-Saxon). This is primarily to provide an example of how to
support a language with multiple scripts (or simply a language with
a non-Latin script).

\StopEventually{%
  \PrintChanges
  \PrintIndex
}
\end{document}
