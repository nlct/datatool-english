% arara: lualatex
% arara: makeindex
% arara: lualatex
% arara: makeindex
% arara: lualatex
% arara: lualatex
\PassOptionsToPackage{noprint}{doc}% command names are too long
\documentclass{ltxdoc}

\usepackage{fontspec}
\usepackage{tcolorbox}
\usepackage[left=2.5cm,right=2.5cm]{geometry}
\usepackage[locales={ang-Runr,ang-Latn,en}]{datatool-base}

\setromanfont{Noto Serif}
\setsansfont{Noto Sans}
\setmonofont{Noto Sans Mono}
\newfontfamily\runefont{Noto Sans Runic}
\NewDocumentCommand{\textrune}{m}{{\runefont #1}}

\CheckSum{0}

\newcommand*{\sty}[1]{\textsf{#1}}
\newcommand*{\file}[1]{\texorpdfstring{\nolinkurl{#1}}{#1}}
\newcommand*{\filemeta}[3]{%
 \texorpdfstring
  {\nolinkurl{#1}\meta{#2}\nolinkurl{#3}}%
  {#1<#2>#3}%
}
\newcommand*{\filemetameta}[5]{%
 \texorpdfstring
 {\nolinkurl{#1}\meta{#2}\nolinkurl{#3}\meta{#4}\nolinkurl{#5}}%
 {#1<#2>#3<#4>#5}%
}
\newcommand*{\opt}[1]{\textsf{#1}}
\newcommand*{\qt}[1]{“#1”}
\newcommand*{\setlocaleopts}[1]{%
 \cs{DTLsetLocaleOptions}\allowbreak
 \texttt{\brackets{\strong{#1}}}\allowbreak
 \marg{key=val list}%
}

\definecolor{defbackground}{rgb}{1,1,0.75}
\newtcolorbox{definition}{colback=defbackground}
\newtcolorbox{important}{colback=red!5!white,colframe=red}

\tcbuselibrary{documentation}
\tcbset{verbatim ignore percent}

\RecordChanges
\PageIndex

\newcommand{\myLatnList}{f, u, þ, o, r, c}
\newcommand{\myRunrList}{
 \textrune{ᚠ},
 \textrune{ᚢ},
 \textrune{ᚦ},
 \textrune{ᚩ}, 
 \textrune{ᚱ}, 
 \textrune{ᚳ}
}
\DTLshufflelist\myLatnList
\DTLshufflelist\myRunrList

\title{English and Old English Localisation Support for \sty{datatool} Package}
\author{Nicola L. C. Talbot}
%%DATECMD%%

\begin{document}
\maketitle

\begin{abstract}
This is the English and Old English localisation support for the \sty{datatool}
package (version 3.0+). This needs to be installed in addition to 
\sty{datatool}. To ensure regional support, you will also need to
install \sty{datatool-regions}.
\end{abstract}

\tableofcontents

\section{Introduction}
\label{sec:intro}

This bundle provides the English and Old English modules for \sty{datatool} v3.0+.
The files simply need to be installed on \TeX's path.
(They will be ignored if a pre-3.0 version of \sty{datatool} is installed.)
The \sty{datatool-base} package (which is automatically loaded by
\sty{datatool}) uses \sty{tracklang}'s interface for detecting
localisation settings and finding the appropriate files.
If you use \sty{babel} or \sty{polyglossia}, make sure that you
specify the document languages before the first package to load
\sty{tracklang}.

For example:
\begin{dispListing}
\usepackage[british]{babel}
\usepackage{datatool-base}
\end{dispListing}
Alternatively, if you are not using a language package, simply use
the \opt{locales} option. For example:
\begin{dispListing}
\usepackage[locales={en-GB}]{datatool-base}
\end{dispListing}
Any option that can be passed to \sty{datatool-base} can also be
passed to \sty{datatool} but if \sty{datatool-base} has already been
loaded, it will be too late to use the \opt{locales} option.
For example:
\begin{dispListing}
\usepackage[locales={en-GB}]{datatool}
\end{dispListing}
But not:
\begin{dispListing*}{title={Incorrect!},colframe=red}
\usepackage{datatool-base}
\usepackage[locales={en-GB}]{datatool}
\end{dispListing*}

If another package that also loads \sty{tracklang} is loaded first,
then \sty{datatool-base} can pick up the settings from that. For
example:
\begin{dispListing}
\usepackage[en-GB]{datetime2}
\usepackage{datatool}
\end{dispListing}

For Old English (Anglo-Saxon), you need the three letter language
code \qt{ang} and the four letter script code: \qt{Latn}
(Latin) or \qt{Runr} (Runic). Old English support is very limited
and mainly provided as an example of how to support extended Latin
and non-Latin languages. See section~\ref{sec:anglosaxon} for
further information.

If the regional file isn't installed or if no region is associated
with the locale then only the language settings will be implemented.
For example:
\begin{dispListing}
\usepackage[english]{babel}
\usepackage{datatool-base}
\end{dispListing}
In this case there is no region so only \file{datatool-english.ldf}
will be loaded.

Supplementary packages provided with \sty{datatool} can also have
the locales provided. For example:
\begin{dispListing}
\documentclass{article}
\usepackage[locales={en}]{datagidx}
\newgidx{index}{Index}
\newterm{élan}
\newterm{elephant}
\newterm{élite}
\newterm{elk}
\begin{document}
\gls{elk}, \gls{élan} \gls{élite} and \gls{elephant}.
\printterms
\end{document}
\end{dispListing}
As with \sty{datatool}, these supplementary packages internally load
\sty{datatool-base} so if that has already been loaded, then the
localisation support should already have been set.

The \sty{glossaries} package also loads \sty{datatool-base}
and, as from \sty{glossaries} version 4.55, it will check for the 
new \sty{datatool-base} commands. If they are defined then 
\cs{printnoidxglossary} will switch to this
new method of sorting otherwise it will fallback on the old method.
So installing \sty{datatool-english} will not only affect
\sty{datatool} and its associated packages but also
\sty{glossaries}. For example:
\begin{dispListing}
\documentclass{article}
\usepackage[locales={en},index,style=indexgroup]{glossaries}
\makenoidxglossaries
\newterm{élan}
\newterm{elephant}
\newterm{élite}
\newterm{elk}
\begin{document}
\gls{elk}, \gls{élan} \gls{élite} and \gls{elephant}.
\printnoidxglossary[type=index]
\end{document}
\end{dispListing}

\begin{docCommand}{DTLenLocaleHook}{}
The \file{datatool-english.ldf} file provides an intermediary command
\cs{DTLenLocaleHook} that's automatically added to the captions hook.
This performs all the \sty{datatool-base} English language redefinitions.
There should be little reason to actually use this command anywhere
unless you're not using a language package with caption hooks.
\end{docCommand}

\section{English (\qt{en})}

The \file{datatool-english.ldf} file provides support for English
orthography (section~\ref{sec:orthography}), parsing
dates and times (experimental, see section~\ref{sec:temporal}),
and for providing fixed text translations.

Aside from the letter group commands, described in
section~\ref{sec:orthography}, the only fixed text for 
\sty{datatool-base} is \cs{DTLandname} and the textual data types
used by \cs{DTLgetDataTypeName}. These are redefined implicitly
via the command \cs{DTLenTranslations}, which allows an alternative
definition of \cs{DTLandname} to be provided without having to add anything to the
caption hook.

Options provided by \file{datatool-english.ldf} may be set with
\setlocaleopts{en}.  Currently, there is only one option.

\begin{docKey}{and}{=\meta{setting}}{initial \opt{word}}
This determines the definition of \cs{DTLandname} within
\cs{DTLenSetAndName} used by
\cs{DTLenTranslations}. The value may be:
\docValue{word} (define \cs{DTLandname} to expand to
\qt{and}) or \docValue{amp} (define \cs{DTLandname}
to expand to \cs{\&}).
\end{docKey}

Aside from \file{datatool-english.ldf}, there is also
\file{databib-english.ldf} to provide localisation support
for \sty{databib} (see section~\ref{sec:databib})
and \file{person-english.ldf} to provide localisation support
for the \sty{person} package (see section~\ref{sec:person}).

\subsection{Orthography}
\label{sec:orthography}

The \file{datatool-english.ldf} file provides support for English
orthography. This deals with how words are sorted according to the
English alphabet. The way that the new \sty{datatool} v3.0 sorting
commands work is to use a handler function that converts the
original content into a byte string. Since there are no byte array
data types in \TeX, this is implemented by converting the original
string into an ASCII string in such a way that sorting by character
code will produce the desired order.

With the newer \cs{DTLsortwordlist}, this processing is performed once for each
item in the list prior to sorting to allow for a faster character code
sort of a temporary sequence variable that stores both the original
(actual value) and the transformed (sort value) item. 
This allows the original item to be restored afterwards.
(You can use \cs{show} on the comma-separate list command afterwards
to double-check this information.)

For example, suppose the document starts with:
\begin{dispListing*}{title={\sty{datatool} v3.0+}}
\documentclass{article}
\usepackage[locales={en}]{datatool}
\end{dispListing*}
\DTLenLocaleHook
If \file{datatool-english.ldf} is installed:
\begin{dispExample*}{title={With Localisation}}
\newcommand{\mylist}{élan,zebra,elephant,ant,élite,elk}
\DTLsortwordlist{\mylist}{\DTLsortwordcasehandler}
Sorted list: \DTLformatlist{\mylist}.
\end{dispExample*}
However, if \file{datatool-english.ldf} isn't installed (or if
English support hasn't been requested) the resulting list would be:
\begingroup
\DTLresetLanguage
\begin{dispExample*}{title={No Localisation}}
\newcommand{\mylist}{élan,zebra,elephant,ant,élite,elk}
\DTLsortwordlist{\mylist}{\DTLsortwordcasehandler}
Sorted list: \DTLformatlist{\mylist}.
\end{dispExample*}
\endgroup
This is because a simple character code comparison has been used.
(Note also the use of the default \& instead of \qt{and}.)

The \cs{DTLsortwordcasehandler} function is for case-sensitive word
sorting, whereas the \cs{DTLsortwordhandler} function is for
case-insensitive word sorting and simply converts each element to
lowercase before processing using the locale handler.  There are
analogous functions for letter-order comparisons which strip spaces
and hyphens before processing (see the \sty{datatool} manual for
further details).

The \cs{DTLsortdata} command, provided by \sty{datatool} v3.0+ for
sorting databases, works in a similar way to \cs{DTLsortwordlist} 
and uses the same handler functions. For example:
\begin{dispListing}
\documentclass{article}
\usepackage[locales={en}]{datatool}

\DTLaction{new}
\DTLaction[ assign = { name = Annie, age = 40 } ]{new row}
\DTLaction[ assign = { name = Ele, age = 80 } ]{new row}
\DTLaction[ assign = { name = Éleanor, age = 28 } ]{new row}
\DTLaction[ assign = { name = Zack, age = 47 } ]{new row}
\DTLaction[ assign = { name = Elva, age = 53 } ]{new row}
\DTLaction[ assign = { name = Æthelwulf, age = 95 } ]{new row}

\DTLsortdata{}{name}
\begin{document}
\DTLaction{display}
\end{document}
\end{dispListing}
If \file{datatool-english.ldf} is installed, then the order will be:
Æthelwulf, Annie, Ele, Éleanor, Elva, Zack.

Without \file{datatool-english.ldf}, the order will be:
Annie, Ele, Elva, Zack, Æthelwulf, Éleanor.

If letter groups are required, an extra column can be added to the
database with the corresponding letter group obtained from sorting:
\begin{dispListing}
\DTLsortdata[save-group]{}{name}
\end{dispListing}
The letter group is typically the first letter of the sort value
or the actual value but this may not be the case for some languages,
so the code for obtaining the first letter of a word is adjusted by
the language hook.

The groups obtained from sorting are letter groups if the sorted
item starts with an alphabetical character. If the value is
determined to be a currency, it will be considered part of the
currency group, if the value is determined to be a number (without a
currency prefix) then it will be considered part of the number
group, otherwise it will be in the non-letter group.

The titles for all these groups are obtained with the
\sty{datatool-base} commands
\cs{dtllettergroup}, \cs{dtlnonlettergroup}, \cs{dtlnumbergroup}
and \cs{dtlcurrencygroup} which are all redefined by the language
hook.

In the case of English, the currency group title simply expands to
\qt{Currency}, the number group title simply expands to
\qt{Numbers}, the non-letter group title simply expands to
\qt{Symbols}.  The letter group title expands to its argument
converted to title case.

If you need to make any adjustments, the following commands are
provided by \file{datatool-english.ldf} for the handler and 
groups.

\begin{docCommand}{DTLenLocaleHandler}{\marg{tl-var}}
Converts the token list variable containing the sort value.
You may redefine this command to add support for additional
characters, but bear in mind that the more complex the regular
expression the longer the document build.
\end{docCommand}

\begin{docCommand}{DTLenLocaleGetGroupString}{\marg{actual}\marg{sort}\marg{tl-var}}
Sets \meta{tl-var} to the applicable content in preparation for
extracting the initial letter, where \meta{actual} is the original
value and \meta{sort} is the sort value obtained after processing 
by the handler. (Afterwards, the expansion of \meta{tl-var} will
then be passed as the \meta{word} argument of \cs{DTLenLocaleGetInitialLetter}.)

For English, this is straight-forward
as the sort value is usually appropriate for this.
If you redefine this command to set \meta{tl-var} to the actual
value then accented characters will be considered separate letter
groups. This can lead to odd results.
\end{docCommand}

\begin{docCommand}{DTLenLocaleGetInitialLetter}{\marg{word}\marg{tl-var}}
Gets the initial letter of \meta{word} and stores it in the given
token list variable.
\end{docCommand}

\begin{docCommand}{DTLenSetLetterGroups}{}
Redefines the group commands. This can be redefined if different
titles are required.
\end{docCommand}

\subsection{Dates and Times}
\label{sec:temporal}

This is still an experimental feature. As from v3.0, \sty{datatool}
now has additional data types: date, time and datetime.
\emph{Parsing for these is \strong{off} by default but may be enabled.}
Without the language support, only ISO dates, times and timestamps
can be parsed. The order (dmy, mdy, ymd) is usually set by the
region file (provided with \sty{datatool-regions}).
For example, \file{datatool-GB.ldf} sets the order to dmy
whereas \file{datatool-US.ldf} sets the order to mdy.
The region files only deal with numeric dates and times but
have a hook to allow supporting language files to extend this to
parsing temporal values that contain textual content, such as month
names.

The following commands are provided. Note that they use
\LaTeX3 syntax, which needs to be enabled.

\begin{docCommand*}{datatool_en_get_monthname_map:n}{ \marg{month}}
Gets the number associated with the given English month name.
For example, January or Jan should result in 1.
\end{docCommand*}

\begin{docCommand*}{datatool_en_if_pm:nTF}{ \marg{text} \marg{true} \marg{false}}
Tests if \meta{text} represents afternoon (\qt{pm}, 
\qt{in the afternoon}, \qt{in the evening}, \qt{midnight}).
This indicates that 12 needs to be added to the hour for 12-hour
formats. Not that \qt{noon} is not \emph{after} noon (12 noon +12 would
result in 24, which is incorrect for noon) and
\qt{midnight} would need to have the hour as 12 not 0 (0 midnight +12 would
result in 12, which is incorrect for midnight).
\end{docCommand*}

\begin{docCommand*}{datatool_en_monthname:n}{ \marg{num}}
Expands to the month name for \meta{num} (1 for January, etc).
\end{docCommand*}

\begin{docCommand*}{datatool_en_shortmonthname:n}{ \marg{num}}
Expands to the abbreviated month name. This will be defined to
\cs{DTMenglishshortmonthname} if that command exists
(\sty{datetime2-english})
or to:
\end{docCommand*}
\begin{docCommand*}{datatool_en_shortmonthname_dotless:n}{ \marg{num}}
Expands to a three-letter abbreviation without a trailing dot.
\end{docCommand*}
An alternative command is also provided:
\begin{docCommand*}{datatool_en_shortmonthname_dotted:n}{ \marg{num}}
This expands to a three or four letter abbreviation followed by a
dot except for short month names, such as May, that aren't
abbreviated.
\end{docCommand*}

\subsection{Support for \sty{databib}}
\label{sec:databib}

The \sty{datatool-english} bundle provides English language support for
the \sty{databib} package. Options may be set with
\setlocaleopts{en/databib}.

Currently, there is only one option:
\begin{docKey}{short-month-style}{=\meta{style}}{initial \opt{dotted}}
This determines the month name abbreviations for 
\sty{databib}'s \opt{abbrv} style.
The value may be \docValue{dotted} (use dotted month
abbreviations) or \docValue{dotless} (use three letter abbreviations
with no dot). 
\end{docKey}

\begin{docCommand}{DataBibEnglish}{}
The \file{databib-english.ldf} file defines the
intermediary command \cs{DataBibEnglish}
that redefines all the fixed text commands. This command
is automatically added to the captions hook if \sty{databib}
is loaded.
\end{docCommand}

\subsection{Support for \sty{person}}
\label{sec:person}

\begin{docCommand}{DataToolPersonEnglish}{}
The \sty{datatool-english} bundle provides English language support for
the \sty{person} package. There are currently no options.
The \file{person-english.ldf} file defines the intermediary
command \cs{DataToolPersonEnglish}
that redefines all the fixed text commands. This command
is automatically added to the captions hook if the \sty{person}
package is loaded.
\end{docCommand}

\section{Old English (Anglo-Saxon, \qt{ang})}
\label{sec:anglosaxon}

This bundle also supplies very limited support for Old English
(Anglo-Saxon). Only UTF-8 encoding is supported. 
This is primarily to provide an example of how to
support a language with multiple scripts (or simply a language with
a non-Latin script). You will need to identify the script as well as
the language. For the Latin script:
\begin{dispListing}
\usepackage[locales={ang-Latn}]{datatool-base}
\end{dispListing}
For the Runic script:
\begin{dispListing}
\usepackage[locales={ang-Runr}]{datatool-base}
\end{dispListing}
Note that the script indicates the script of the input or source
text. That is, the text used in the document source code, which
may not correspond to the glyphs visible in the PDF file.

For example, suppose a package provides a command called, say \cs{runic},
which expects Latin characters in the argument but the font encoding
ensures that those characters appear as runes in the PDF (for
example, the hypothetical command
\texttt{\cs{runic}\brackets{fuþorc}} in the source code would be
rendered as \textrune{ᚠᚢᚦᚩᚱᚳ} in the PDF). In this case, the source
is Latin and so \qt{ang-Latn} is needed when specifying the locale.

If, however, the source code actually contains characters from  
the Runic Unicode block (with an appropriate font that supports
those characters), then the source is Runic and so \qt{ang-Runr}
is needed when specifying the locale.

If there is no captions hook, the last locale to be tracked is the
one that will be in effect. You can switch using the following:
\begin{docCommand}{DTLangLatnLocaleHook}{}
Switch to \qt{ang-Latn} (if \file{datatool-ang-Latn.ldf} has been
loaded, undefined otherwise).
\end{docCommand}
\begin{docCommand}{DTLangRunrLocaleHook}{}
Switch to \qt{ang-Runr} (if \file{datatool-ang-Runr.ldf} has been
loaded, undefined otherwise).
\end{docCommand}

The \qt{ang-Latn} localisation support provides the following
option, which may be set using
\setlocaleopts{ang-Latn}.

\begin{docKey}[ang-Latn]{order}{=\meta{setting}}{initially \opt{harley}}
Determines the letter order. The \meta{setting} may be:
\docValue{harley} (A-Z, Ƿ, Ð, Æ, Þ), \docValue{stowe}
(ending with Ƿ, Ð, Þ), \docValue{titus}
(ending with Ƿ, Þ, Ð) or \docValue{futhorc} (starts f, u, þ, o, r, c).
\end{docKey}

The \qt{ang-Runr} localisation support provides the following
option, which may be set using
\setlocaleopts{ang-Runr}.

\begin{docKey}[ang-Runr]{order}{=\meta{setting}}{initially \opt{punc-futhorc}}
Determines the rune order. The \meta{setting} may be:
\docValue{punc-unicode} (punctuation first, followed by Unicode order
for the rest),
\docValue{punc-futhorc} (Old English Rune Poem order, punctuation
first), or
\docValue{futhorc-punc} (Old English Rune Poem order, punctuation
last).
Note that the \qt{punc-unicode} order puts the punctuation first,
followed by the other runes, even though the punctuation runes
actually come towards the end of the Runic block.  The Old English
Rune Poem order is fuþorc order, with the additional non-poem runes
appended at the end.
The Tironian et (which isn't a rune) is added to the end of the
non-punctuation runes.
\end{docKey}

Both ang-Latn and ang-Runr define \cs{DTLandname} to the Tironian
et (⁊).
This is done via the following hooks:
\begin{docCommand}{DTLangLatnTranslations}{}
This is defined in \file{datatool-ang-Latn.ldf} to update the
definition of \cs{DTLandname} and is implemented by
\cs{DTLangLatnLocaleHook}.
\end{docCommand}
\begin{docCommand}{DTLangRunrTranslations}{}
This is defined in \file{datatool-ang-Runr.ldf} to update the
definition of \cs{DTLandname} and is implemented by
\cs{DTLangRunrLocaleHook}.
\end{docCommand}
Ideally, the language file should also redefine the commands
\cs{DTLdatatypeunsetname}, \cs{DTLdatatypestringname},
\cs{DTLdatatypeintegername}, \cs{DTLdatatypedecimalname},
\cs{DTLdatatypecurrencyname}, \cs{DTLdatatypedatetimename},
\cs{TLdatatypedatename}, \cs{DTLdatatypetimename}, and
\cs{DTLdatatypeinvalidname}, although these are mostly provided for
debugging.

There is no other localisation support (that is, no other
fixed text or date/time support). The remaining language-sensitive
commands are reset with \cs{DTLresetLanguage} back to their initial
definitions.

Suppose the document uses \sty{fontspec} with fonts that support
the required characters, as follows:
\begin{dispListing}
\usepackage{fontspec}
\setromanfont{Noto Serif}
\setsansfont{Noto Sans}
\setmonofont{Noto Sans Mono}
\newfontfamily\runefont{Noto Sans Runic}
\NewDocumentCommand{\textrune}{m}{{\runefont #1}}
\end{dispListing}
and \sty{datatool-base} has been loaded with support for
both Latin and Runic Anglo-Saxon:
\begin{dispListing}
\usepackage[locales={ang-Runr,ang-Latn}]{datatool-base}
\end{dispListing}
and suppose the command \cs{myLatnList}
is defined to expand to the comma-separated list
\texttt{\myLatnList}
and \cs{myRunrList} 
is defined to expand to the comma-separated list
\texttt{\def\textrune#1{\allowbreak\cs{textrune}\brackets{{\runefont #1}}}\myRunrList}
then these lists can be sorted in the document
as follows:
\begin{dispExample}
\DTLangLatnLocaleHook
\DTLsetLocaleOptions{ang-Latn}{order=harley}
\DTLsortwordlist{\myLatnList}{\DTLsortletterhandler}
Sorted list (Harley): \DTLformatlist{\myLatnList}.

\DTLsetLocaleOptions{ang-Latn}{order=futhorc}
\DTLsortwordlist{\myLatnList}{\DTLsortletterhandler}
Sorted list (fuþorc): \DTLformatlist{\myLatnList}.

\DTLangRunrLocaleHook
\DTLsetLocaleOptions{ang-Runr}{order=futhorc-punc}
\DTLsortwordlist{\myRunrList}{\DTLsortwordhandler}
Sorted list: \DTLformatlist{\myRunrList}.
\end{dispExample}
Note that the letter handler is best with \qt{ang-Latn} as there are
many Anglo-Saxon words with hyphens. (Bosworth Toller's Anglo-Saxon
dictionary ignores hyphens in the order. For example,
\qt{á-ǽlan} comes between \qt{aad} and \qt{áǽðan}.)

\StopEventually{%
  \PrintChanges
  \PrintIndex
}
\end{document}
